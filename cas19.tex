\documentclass[11pt]{article}
\usepackage[margin=1.0in]{geometry}
\usepackage[utf8]{inputenc}
\usepackage{graphicx}
\usepackage{hyperref}
%\usepackage{svg}

%\usepackage{textcomp}
%\usepackage[dvipsnames]{xcolor}
%\usepackage{booktabs}
%\usepackage{soul}
\usepackage{subcaption}
%\captionsetup{compatibility=false}

%\usepackage{cleveref}
%\setcounter{secnumdepth}{0}

\title{Topological Data Analysis of Airway Trees in Chronic Obstructive Pulmonary Disease (COPD)}

%\author{CS 452/652/752 Advanced Algorithms and Applications}
\date{}

\begin{document}

\maketitle

\clearpage
\section{Project Summary}

\subsection{Overview}
The goal of this project is to develop analysis and visualization techniques leveraging topological methods to understand Chronic obstructive pulmonary disease (COPD). COPD is an inflammatory lung disease affecting 24 million individuals in the United States, and is the third leading cause of death. 
%It is characterized by progressive airflow obstruction that is not completely reversible. 
The disease is associated with structural changes to lung tissue and airways. Recent advances in computed tomography (CT) imaging have enabled extensive phenotyping of COPD by allowing morphologic characterization of lung tissue and airways. Current airway measurements describe the morphology of individual branches and assumes that the airway tree geometry follows Euclidean relationships that can be represented in integer dimensions. However, the airway tree possesses a complex self-repeating geometry that is specific for each individual. Topological techniques provide a strong theoretical basis for simplifying and summarizing complex data while still preserving critical underlying structures. We want to leverage two different topological data analysis techniques, persistent homology and contour trees to analyze and extract features from the complex CT data. The ultimate goal is to extract topology based radionomic metrics that can be used to classify CT scans based on different stages of severity of COPD.



\subsection{Intellectual Merit}
The scientific challenges this project addresses are two-fold: how to use topology to extract features from the data; and how to design effective visualizations to communicate these features to domain experts and decision makers. Topological techniques central to this project provide a strong theoretical basis for simplifying and summarizing complex data while still preserving critical underlying structures. They also provide a basis for task-oriented designs that allow us to control the volume of data to be displayed in visualizations, so users can develop faithful mental models of the data, facilitating information discovery. This project focuses on two research agendas. First, it proposes a rich body of topological summarization techniques to extract and preserve important topological features and obtain compact and hierarchical representations that are suitable for visual exploration. The feature extracting process captures complex interactions in the system, describes features at all scales, is robust with respect to noise, and has efficient computation. Second, ....
%this project proposes designing visualizations that encode the extracted topological structures explicitly, focusing on investigating techniques to fully exploit their properties in the visual metaphors to be developed. This project web site provides additional information and will include access to developed tools and test data sets.

\subsection{Broader impacts}
COPD is primarily caused by cigarette smoking and more than half the patients report that COPD limits their ability to work, perform normal physical exertion, and engage in social activities. 
The healthcare costs associated with COPD are $50$ billion, including $30$ billion direct healthcare costs. The burden is expected to rise at a rapid pace because of continued exposure to risk factor and aging of the population.

\clearpage


\section{Team}
Dr Sidharth Kumar \\
Dr sandeep bodduluri \\
ke Fan (PhD Student) \\
Grad student \\
\clearpage

\section{Proposed Research}

\subsection{Persistent Homology}
What is PH ?\\
How PH can be used ?\\

\subsection{Contour tree}
What are contour trees ?\\
How can CT be used ?\\

\subsection{Data processing pipeline}
CT scan $-->$ processing $-->$ Topological analysis and visualization $-->$ insight\\


\clearpage
\section{Anticipated Outcome}
\subsection{radiomic feature}
Lobes/cavity using PH\\
Connected component analysis using PH\\
Ring structure analysis using PH\\


\clearpage
\subsection{Deliverables}


\subsection{Future direction}
This research is going to lay foundation for a long term collaboration between the computer science and the XXXX department. We are trying to solve some of the very basic problems faced in the field of imaging. The ultimate goal is to develop a suit of techniques that can be universally adopted by imaging experts from fields such as  X, Y, Z. To this end, we want to revamp the existing inefficient manual data movement and analysis framework to an fully automiatic feedback driven framework that can not only aid clinicians make better sense of the data. This research will enable the first step towards building a long lasting collaboration.

Scholarly activities and technological impacts that are
expected at the successful completion of the project. How the interdisciplinary
team will proceed after the initial seed funding to sustain the project and seek
extramural support. Also, describe the commercialization potential for this
interdisciplinary project if applicable. The prospect of follow-on extramural
support for the project will be an important review criterion (2 pages maximum).
\clearpage
\section{Budget and budget justification}
The budget and budget justification should cover a support period of one year (2 pages maximum). The budgeting of PI’s and co-PI’s salaries or release time is discouraged.  

\end{document}


